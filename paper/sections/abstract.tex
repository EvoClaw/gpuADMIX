% Bioinformatics structured abstract: Motivation / Results / Availability

\textbf{Motivation:}
Model-based admixture estimation methods such as ADMIXTURE and fastmixture
are widely used to infer individual ancestry proportions from genome-wide
genotype data, but their CPU-bound runtime makes $K$ sweeps with multiple
random seeds impractical at biobank scale.
Existing GPU-accelerated alternatives sacrifice the exact binomial likelihood
model for speed, reducing the accuracy and interpretability of ancestry
estimates.

\textbf{Results:}
We present \gpu{}, which reformulates both the E-step and M-step of the
admixture expectation-maximisation algorithm as GPU-native dense matrix
multiplications, preserving the exact binomial likelihood \emph{model} while
achieving $41\times$ and $213\times$ speedups over fastmixture and ADMIXTURE
on the 1000 Genomes Phase\,3 dataset ($N=3{,}202$; $K=5$).
Three algorithmic innovations amplify these gains: Nesterov momentum reduces
EM iterations by $2.3\times$ and improves converged log-likelihood by $7{,}865$
units over plain EM; stochastic mini-batch EM improves solution quality
while reducing peak GPU memory; and streaming randomised SVD provides
efficient spectral initialisation for large datasets.
The best of five parallel \gpu{} seeds matches or exceeds fastmixture at
every tested $K \in \{2,\ldots,10\}$, while completing five seeds costs less
wall time than a single fastmixture run, making multi-seed inference the
practical default workflow.
We also provide CLUMPAK-lite (\clite{}) for label-consistent
ancestry-proportion visualisation across $K$ values, and a multi-GPU
dispatcher that completes $K=2$--$10$ scans in ${\approx}130$\,s on eight GPUs.

\textbf{Availability and implementation:}
\gpu{} is implemented in Python using PyTorch and is freely available at
\url{https://github.com/[REPO]} under the MIT licence.
