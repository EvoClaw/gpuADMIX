\section{Results}
\label{sec:results}

\gpu{} was benchmarked against \admix{} \citep{alexander2009fast} and
\fm{} \citep{meisner2024fast} on the 1000 Genomes Project Phase\,3 dataset
\citep{1000genomes2015global} comprising 3,202 individuals genotyped at
200,000 LD-pruned autosomal SNPs.
All methods processed the same pre-processed dataset.
\gpu{} ran on a single NVIDIA L20 GPU (48\,GB VRAM); \fm{} and \admix{}
ran on a 32-core Intel Xeon Platinum 8375C server.
Speedups therefore reflect the combined advantage of GPU hardware and the
GPU-native EM design.
Accuracy was assessed via the log-likelihood of the converged solution and
the mean per-component Pearson~$r^2$ between each method's admixture
proportion matrix~$Q$ and that of \admix{} at the same~$K$, after optimal
column alignment via the Hungarian algorithm (\clite{}).

%-------------------------------------------------------------------
\subsection{Speed}
\label{sec:results:speed}

Table~\ref{tab:main} summarises wall time and accuracy at $K=5$.
% Table 1: Main performance comparison at K=5
\begin{table*}[t]
\centering
\caption{%
  Performance at $K=5$ on the 1000 Genomes Phase\,3 dataset
  (3,202 individuals, 200K LD-pruned SNPs).
  Wall time: mean $\pm$ s.d.\ across five runs (\gpu{}, \fm{})
  or a single run (\admix{}, runtime-limited).
  $Q$~$r^2$: mean per-component Pearson~$r^2$ vs \admix{}~$Q$
  after Hungarian alignment.
  Speedup relative to \admix{}.
  Hardware: NVIDIA L20 GPU (\gpu{});
  Intel Xeon Platinum 8375C 32-thread CPU (\fm{}, \admix{}).%
}
\label{tab:main}
\resizebox{\textwidth}{!}{%
\begin{tabular}{lrrrr}
  \toprule
  Method & Wall time (s) & Speedup vs \admix{} & Log-likelihood & $Q$ $r^2$ vs \admix{} \\
  \midrule
  \admix{} & 3{,}583        & $1\times$            & $-241{,}227{,}839$           & 1.000000 \\
  \fm{}     & $694 \pm 40$  & $5\times$            & $-241{,}227{,}643 \pm 0.3$   & 0.999984 \\
  \gpu{}    & $\mathbf{16.8 \pm 3.4}$ & $\mathbf{213\times}$ &
              $\mathbf{-241{,}224{,}751 \pm 98}$ & $\mathbf{0.999987}$ \\
  \bottomrule
\end{tabular}%
}
\end{table*}

\gpu{} converges in $16.8 \pm 3.4$\,s (mean $\pm$ s.d., five independent
seeds), compared with $694 \pm 40$\,s for \fm{} (five seeds) and 3,583\,s
for \admix{} (single run; replicate runs were infeasible at this scale).
This yields $41\times$ and $213\times$ speedups over \fm{} and \admix{},
respectively.
Across the full $K=2$--$10$ scan, \gpu{} wall time remains below 60\,s for
every $K$ tested (Figure~\ref{fig:kscan}d).
Critically, running \gpu{} with five independent seeds at $K=5$ costs
$\approx\!84$\,s in total—comparable to a single \fm{} run—so multi-seed
inference becomes routine on GPU precisely where it would be prohibitive on CPU.

%-------------------------------------------------------------------
\subsection{Accuracy}
\label{sec:results:accuracy}

Despite the hardware-accelerated speedup, \gpu{} matches or exceeds both
baselines in solution quality (Table~\ref{tab:main}).
At $K=5$, the mean log-likelihood across five seeds is
$-241{,}224{,}751 \pm 98$, an improvement of 3,088 units over the single
\admix{} run and 2,892 units over the \fm{} mean.
The admixture proportion matrices are virtually identical to those of \admix{}
($Q$\,$r^2 = 0.999987$ vs \admix{}~$Q$, mean over five \gpu{} seeds),
confirming that neither the GPU-native reformulation nor the stochastic
mini-batch updates compromise estimation fidelity.

Welch's two-sample $t$-test on per-seed log-likelihoods
($n_{\gpu{}} = 5$, $n_{\fm{}} = 5$, $\mathrm{df} \approx 5.0$) confirms
that the \gpu{} advantage over \fm{} is statistically significant
($t_{5.0} = 18.3$, $p < 10^{-4}$).
Within \gpu{}, FISTA-style Nesterov momentum significantly outperforms
plain EM at matched seeds ($t_{7.8} = 42.1$, $p < 10^{-6}$),
demonstrating that momentum improves both convergence speed and final
solution quality.

%-------------------------------------------------------------------
\subsection{$K$ scan and multi-seed strategy at high~$K$}
\label{sec:results:kscan}

\begin{figure}[t]
  \centering
  \includegraphics[width=\linewidth]{figures/fig2_k_scan.pdf}
  \caption{%
    \textbf{$K$ scan results.}
    (a)~$\Delta$Log-likelihood of \gpu{} vs \fm{} across $K=2$--$10$ (mean
    and best-of-five seeds shown as bars; positive = better than \fm{}).
    (b)~Run-stability RMSE across five seeds per~$K$.
    (c)~5-fold cross-validation log-likelihood per~$K$.
    (d)~Wall time per $K$ value for \gpu{} (mean $\pm$ s.d., five seeds).%
  }
  \label{fig:kscan}
\end{figure}

Across $K=2$--$10$, \gpu{} achieves comparable or better log-likelihood than
\fm{} when the best-of-five seed is considered (Figure~\ref{fig:kscan}a).
For $K \leq 7$, the \gpu{} per-seed mean already equals or exceeds the
\fm{} mean across five seeds.
At $K \geq 8$, the EM objective landscape becomes increasingly multimodal:
the \gpu{} mean log-likelihood falls slightly below the \fm{} mean at $K=8$
and $K=9$, reflecting occasional convergence to suboptimal local optima.
The best-of-five \gpu{} seed nonetheless matches or exceeds the best \fm{}
seed at every~$K$, including a $+45{,}663$-unit advantage at $K=10$.
Because five \gpu{} seeds at $K=10$ complete in $\approx\!300$\,s—roughly
the same total compute as a single \fm{} run—the multi-seed strategy is
practically justified precisely when the landscape is most challenging.

The CLUMPAK-lite run-stability diagnostic corroborates this picture
(Figure~\ref{fig:kscan}b): the mean pairwise RMSE (in admixture proportion
units, $0$--$1$) across five seeds is below 0.02 for $K \leq 7$ and rises
to approximately 0.04 at $K=9$--$10$, indicating greater solution variability
but not instability in the biological interpretation.

%-------------------------------------------------------------------
\subsection{$K$ selection by cross-validation}
\label{sec:results:cv}

Five-fold SNP hold-out cross-validation
(Section~\ref{sec:methods:cv}) provides a data-driven complement to
information criteria (Figure~\ref{fig:kscan}c).
The held-out log-likelihood reaches its global maximum at $K=9$
($-240{,}873{,}512$); the largest per-$K$ improvement occurs at $K=5$,
consistent with the five major continental ancestry groups in the dataset.
The BIC independently reaches its minimum at $K=4$, favouring the most
parsimonious partition of the data.
The CV optimum at $K=9$ is accessible only via a multi-seed strategy owing
to the multimodal landscape at high~$K$, reinforcing the practical value
of \gpu{}'s speed in this regime.

%-------------------------------------------------------------------
\subsection{Ablation study}
\label{sec:results:ablation}

% Table 2: Ablation study at K=5
\begin{table*}[t]
\centering
\caption{%
  Ablation study at $K=5$, averaged over five seeds.
  Each variant removes one component while holding all other settings fixed.
  $\Delta$LL is relative to the full \gpu{} model.%
}
\label{tab:ablation}
\resizebox{0.85\textwidth}{!}{%
\begin{tabular}{lrrrr}
  \toprule
  Variant & Wall time (s) & Iterations & Log-likelihood & $\Delta$LL \\
  \midrule
  \gpu{} (full)               & $16.8 \pm 3.4$ & $47 \pm 8$   & $-241{,}224{,}751 \pm 98$  & 0 \\
  \quad $-$ Nesterov momentum & $19.6 \pm 2.5$ & $107 \pm 11$ & $-241{,}232{,}616 \pm 71$  & $-7{,}865$ \\
  \quad $-$ Mini-batch EM     & $24.4 \pm 3.1$ & $64 \pm 9$   & $-241{,}225{,}502 \pm 114$ & $-751$ \\
  \quad $-$ SVD init          & $18.2 \pm 4.8$ & $83 \pm 15$  & $-241{,}227{,}384 \pm 203$ & $-2{,}633$ \\
  \bottomrule
\end{tabular}%
}
\end{table*}


Table~\ref{tab:ablation} quantifies the contribution of each algorithmic
component at $K=5$.

\paragraph{Nesterov momentum.}
Removing the FISTA-style momentum (Section~\ref{sec:nesterov}) increases
iterations from $47 \pm 8$ to $107 \pm 11$ ($2.3\times$) and lowers the
converged log-likelihood by 7,865 units (${\approx}0.003\%$ of total LL
magnitude), the largest single ablation penalty.
The joint degradation in iteration count and solution quality indicates that
momentum assists escape from shallow local optima in addition to accelerating
convergence.

\paragraph{Mini-batch EM.}
Disabling stochastic SNP partitioning increases wall time by $45\%$
(from 16.8 to $24.4$\,s) and marginally reduces solution quality
($\Delta\mathrm{LL} = -751$; ${\approx}3{\times}10^{-4}\%$).

\paragraph{SVD initialisation.}
Replacing the streaming randomised SVD with random Dirichlet initialisation
increases iterations from $47$ to $83$ ($1.8\times$) and lowers the
final log-likelihood by 2,633 units (${\approx}0.001\%$), confirming that
spectral initialisation provides a substantially better starting point.

%-------------------------------------------------------------------
\subsection{Population structure visualisation}
\label{sec:results:structure}

\begin{figure}[t]
  \centering
  \includegraphics[width=\linewidth]{figures/fig4_structure_plot.pdf}
  \caption{%
    Admixture bar plot (STRUCTURE-style) for $K=2$--$7$ produced by
    \clite{} on the 1000 Genomes Phase\,3 dataset.
    Individuals are sorted by super-population label (AFR, AMR, EAS, EUR, SAS).
    Components are aligned within and across $K$ via the Hungarian algorithm
    and greedy bottom-up procedure, respectively.%
  }
  \label{fig:structure}
\end{figure}

Figure~\ref{fig:structure} shows the admixture bar plot for $K=2$--$7$
produced by \clite{}.
The five continental clusters (AFR, AMR, EAS, EUR, SAS) emerge cleanly at
$K=5$ and remain stable as $K$ increases, with each additional component
capturing recognisable sub-continental differentiation.
Run-RMSE below 0.02 across $K=2$--$7$ (Figure~\ref{fig:kscan}b) confirms
that the displayed solution is representative of the inferred distribution
rather than an artefact of a single seed.

%-------------------------------------------------------------------
Collectively, these results demonstrate that \gpu{} achieves one to two
orders of magnitude faster inference than state-of-the-art CPU tools while
matching or exceeding their accuracy, that a GPU-enabled multi-seed strategy
extends this advantage to the multimodal high-$K$ regime, and that all three
core algorithmic components contribute substantively to performance.
