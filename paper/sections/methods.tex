\section{Methods}
\label{sec:methods}

\subsection{Probabilistic model}

We adopt the binomial admixture model of \citet{alexander2009fast}.
Let $\mathbf{G} \in \{0,1,2\}^{N \times M}$ denote the genotype matrix for
$N$ individuals at $M$ biallelic SNPs, where $G_{ij}$ counts the number of
copies of the alternate allele.
The model posits $K$ ancestral populations characterised by two
parameter matrices:
the admixture proportion matrix $\mathbf{Q} \in [0,1]^{N \times K}$
($\sum_k Q_{ik} = 1$ for all $i$) and the allele-frequency matrix
$\mathbf{P} \in (0,1)^{M \times K}$.
Under Hardy--Weinberg equilibrium within each ancestral population,
the marginal genotype likelihood at locus $j$ for individual $i$ is

\begin{equation}
  P(G_{ij} \mid \mathbf{q}_i, \mathbf{p}_j)
  = \binom{2}{G_{ij}} H_{ij}^{G_{ij}} (1 - H_{ij})^{2-G_{ij}},
  \label{eq:likelihood}
\end{equation}

where $H_{ij} = \sum_k Q_{ik} P_{jk} = (\mathbf{Q}\mathbf{P}^{\top})_{ij}$
is the expected frequency of the alternate allele in individual $i$.
Summing over all loci and individuals (omitting constant combinatorial terms)
yields the log-likelihood objective:

\begin{equation}
  \mathcal{L}(\mathbf{Q}, \mathbf{P})
  = \sum_{i=1}^{N} \sum_{j=1}^{M}
    \bigl[ G_{ij} \log H_{ij} + (2 - G_{ij}) \log(1 - H_{ij}) \bigr].
  \label{eq:loglik}
\end{equation}

\noindent Both $\mathbf{Q}$ and $\mathbf{P}$ are estimated by
maximum likelihood via the EM algorithm \citep{dempster1977maximum}.

\subsection{GPU-native EM via matrix reformulation}

The standard EM update for the admixture model \citep{alexander2009fast}
involves per-individual and per-variant summations that are expressed here as
dense matrix operations, making them directly amenable to GPU acceleration.

\paragraph{E-step.}
Given current parameters $(\mathbf{Q}^{(t)}, \mathbf{P}^{(t)})$, compute
the mixture-frequency matrix $\mathbf{H} = \mathbf{Q}\mathbf{P}^{\top} \in \mathbb{R}^{N\times M}$
and the fractional responsibilities

\begin{equation}
  \mathbf{R}^{-} = \frac{\mathbf{G}}{\mathbf{H}}, \qquad
  \mathbf{R}^{+} = \frac{2 \cdot \mathbf{1} - \mathbf{G}}{\mathbf{1} - \mathbf{H}},
\end{equation}

where division is elementwise. $R^{-}_{ij}$ and $R^{+}_{ij}$ represent the
expected contributions of minor and major alleles to individual $i$ at locus $j$.

\paragraph{M-step.}
The complete-data sufficient statistics factorize into two
GEMM (General Matrix Multiply) operations:

\begin{align}
  \mathbf{P}^{(t+1)} &\propto \mathbf{P}^{(t)} \odot
    \frac{(\mathbf{R}^{-})^{\top}\mathbf{Q}^{(t)}}
         {2\,\mathbf{1}_{M}^{\top}\mathbf{Q}^{(t)}},
  \label{eq:p_update} \\[4pt]
  \mathbf{Q}^{(t+1)} &\propto \mathbf{Q}^{(t)} \odot
    \bigl(\mathbf{R}^{-}\mathbf{P}^{(t)} + \mathbf{R}^{+}(\mathbf{1}-\mathbf{P}^{(t)})\bigr),
  \label{eq:q_update}
\end{align}

where $\odot$ is elementwise multiplication and $\propto$ denotes
row-normalisation (for $\mathbf{Q}$) or clipping to $(0,1)$ (for $\mathbf{P}$).
The dominant cost is the three dense matrix products
$\mathbf{Q}\mathbf{P}^{\top}$, $(\mathbf{R}^{-})^{\top}\mathbf{Q}$, and
$\mathbf{R}^{-}\mathbf{P}$, each computable in $O(NMK)$ FLOPS.
On a GPU, these operations map directly to cuBLAS \texttt{SGEMM} calls,
which achieve near-peak throughput for large $N$, $M$, and $K$.
All matrices are stored as 32-bit floating-point tensors in GPU VRAM,
and implemented using PyTorch \citep{paszke2019pytorch} for portability
across GPU architectures.

\subsection{Nesterov momentum acceleration}
\label{sec:nesterov}

Vanilla EM is monotone-increasing but can converge slowly near saddle points.
We incorporate Nesterov momentum \citep{nesterov1983method} directly in the
iterate space of $(\mathbf{Q}, \mathbf{P})$, drawing on a line of work
connecting first-order acceleration to EM \citep{varadhan2008simple}.

Before each E--M step pair, we form the \emph{extrapolated iterates}

\begin{equation}
  \tilde{\mathbf{Q}}^{(t)} = \mathbf{Q}^{(t)} + \alpha_t
    \bigl(\mathbf{Q}^{(t)} - \mathbf{Q}^{(t-1)}\bigr), \qquad
  \tilde{\mathbf{P}}^{(t)} = \mathbf{P}^{(t)} + \alpha_t
    \bigl(\mathbf{P}^{(t)} - \mathbf{P}^{(t-1)}\bigr),
  \label{eq:nesterov}
\end{equation}

with the Nesterov coefficient $\alpha_t = (t-1)/(t+2)$,
and then apply the EM update from $(\tilde{\mathbf{Q}}^{(t)},\tilde{\mathbf{P}}^{(t)})$.
After extrapolation, $\tilde{\mathbf{Q}}$ is projected onto the probability
simplex and $\tilde{\mathbf{P}}$ is clipped to $(10^{-6}, 1-10^{-6})$
to maintain valid parameters.
If the extrapolated iterate decreases the observed log-likelihood relative to
the current iterate, the step is reset to the non-extrapolated EM update
($\alpha_t \leftarrow 0$), preserving the monotone ascent guarantee for
that iteration.

\subsection{Stochastic mini-batch EM}

To accelerate per-epoch computation and improve exploration of the likelihood
landscape, we partition the $M$ SNPs into $B$ random mini-batches per epoch.
In each epoch, the $B$ batches are processed sequentially: for each batch
of $M/B$ SNPs, a full E--M pass updates both $\mathbf{Q}$ and $\mathbf{P}$
using only those SNPs.
Because $\mathbf{Q}$ is updated $B$ times per epoch (once per batch),
convergence requires fewer epochs than full-batch EM.
In practice, mini-batch noise acts as an implicit perturbation that helps
escape shallow local optima in the admixture likelihood landscape.
The default batch count $B$ is set to $\lfloor M / 12{,}500 \rfloor$
based on a grid search over the 1000 Genomes Project dataset.

\subsection{Streaming randomised SVD initialisation}

A principled starting point is critical for EM in the admixture model
\citep{meisner2024fast}.
We initialise $(\mathbf{Q}, \mathbf{P})$ from a rank-$K$ approximation of the
centred genotype matrix $\mathbf{G}_c = \mathbf{G} - 2\hat{\mathbf{p}}\mathbf{1}^{\top}$
(where $\hat{p}_j = \overline{G}_{\cdot j}/2$ is the sample minor allele frequency),
using the randomised SVD algorithm of \citet{halko2011finding}.
For large $N$, forming $\mathbf{G}_c\mathbf{G}_c^{\top}$ directly would require
$O(N^2 M)$ FLOPS and $O(N^2)$ memory.
Instead, we process $\mathbf{G}_c$ in column blocks of fixed size,
streaming the block Gram contributions into an accumulator without ever
materialising the full outer-product matrix.
The resulting compact $K \times K$ factor is then decomposed by standard SVD
to yield the initialisation.
$\mathbf{Q}^{(0)}$ and $\mathbf{P}^{(0)}$ are constructed from the leading
singular vectors and projected onto their respective feasible sets.

\subsection{CLUMPAK-lite: cross-run label alignment}

A known complication of EM-based ancestry inference is label switching: the
$K!$ permutations of ancestral population indices produce equivalent likelihood
values, so independent runs with different random seeds may label the same
ancestry component differently \citep{jakobsson2007clumpp, kopelman2015clumpak}.
We implement \textsc{CLUMPAK-lite}, a pure-Python label-alignment module that
resolves this in two stages.

\paragraph{Within-$K$ alignment.}
Given $S$ independent runs at the same $K$, one run is designated reference.
All others are aligned to it by finding the column permutation $\pi$ of
$\mathbf{Q}_s$ (and identically of $\mathbf{P}_s$) that maximises the sum of
pairwise Pearson correlations between aligned column pairs.
This is equivalent to a maximum-weight bipartite matching and is solved exactly
using the Hungarian algorithm \citep{kuhn1955hungarian} in $O(K^3)$ time.
The across-seed consistency is quantified by the within-$K$ RMSE between
all aligned $\mathbf{Q}$ matrices and their centroid,
serving as an empirical multimodality diagnostic:
high RMSE indicates genuinely distinct local optima, while low RMSE
confirms that all seeds converged to the same basin.

\paragraph{Across-$K$ alignment.}
To produce coherent structure plots across increasing $K$, ancestral components
are aligned bottom-up from $K=2$ to $K=K_{\max}$ by greedy matching:
at each step, the $K-1$ components of the aligned $K$-solution are matched to
the nearest column of the $K$-solution using Pearson correlation as the
similarity metric.
The unmatched column represents the novel component introduced at that $K$.
This procedure ensures that components representing the same ancestry cluster
retain consistent colour and position across panels of the structure plot.

\subsection{Multi-GPU parallel K selection}

Selecting the optimal number of ancestral populations $K$ typically requires
running the model for several values of $K$ and evaluating model-fit criteria
such as the Bayesian Information Criterion \citep{schwarz1978estimating}
or the cross-validation error of \citet{alexander2011genomics}.
Because each value of $K$ is an independent optimisation problem,
gpuADMIX dispatches the $K = 2, \ldots, K_{\max}$ runs in parallel across
all available GPUs using Python's \texttt{multiprocessing} module,
with each process pinned to a dedicated device via \texttt{torch.cuda.set\_device()}.
On an 8-GPU server, the full $K = 2$--$10$ sweep at five random seeds each
completes in approximately 130\,s—a 5.3$\times$ speedup over serial execution.

\subsection{Cross-validation for K selection}
\label{sec:methods:cv}

We implement a 5-fold SNP hold-out cross-validation to provide a data-driven,
model-free estimate of optimal $K$.
SNPs are randomly partitioned into five folds; for each fold the model is
trained on the remaining 80\% of SNPs and the admixture proportions $\mathbf{Q}$
are used as fixed features to estimate allele frequencies $\mathbf{P}_{\text{test}}$
for the held-out 20\% of SNPs via 30 iterations of the M-step with $\mathbf{Q}$ frozen.
The cross-validation score for a given $K$ is the mean hold-out log-likelihood
across all five folds; the optimal $K$ maximises this score.

\subsection{Implementation}

\gpu{} is implemented in Python using PyTorch~1.13+ for GPU tensor operations
and supports PLINK BED format \citep{purcell2007plink} natively via a
memory-efficient bit-unpacking reader.
All benchmarks were performed on an NVIDIA L20 GPU (48\,GB VRAM) for
\gpu{} and an Intel Xeon Platinum 8375C CPU (32 threads) for \fm{} and
\admix{}.
Software and reproducibility scripts are available at
\url{https://github.com/[REPO]}.
